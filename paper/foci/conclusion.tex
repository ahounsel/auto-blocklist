
\section{Conclusion}
We built a block list of 757 censored websites in China that are not
present on the largest block
list~\cite{darer2017filteredweb}. Furthermore, our list is 8.5$\times$
larger than the most widely used block
list~\cite{citizenlab:block}. It contains human rights organizations,
minority news outlets, religious blogs, political dissent groups,
privacy-enhancing technology providers, and more. We've open-sourced
our source code and block list on GitHub with the GPL
license~\cite{censorsearch-lists}.

Automatically detecting which websites are censored in a given country is an
open problem. Systems like CensorSeeker and FilteredWeb are both steps in the
right direction, but there is more work to be done.  One way of improving
CensorSeeker would be to experiment with advanced natural-language processing
techniques to identify better search terms. For example, we could try using
Stanford CoreNLP's part-of-speech tagger to identify phrases that describe
some action against the Chinese government. This approach might
identify the following phrase: ``Chinese citizens protest against the Communist Party on June
4th''. We believe that using such culture-specific phrases as search terms
would enable us to discover even more websites that are censored in China.

% We could also try running CensorSeeker against different search engines to see
% if we get different result sets. It might be the case that one search engine
% is able to find more censored websites than the others. It might also be the
% case that one search engine is better at finding culture-specific websites
% than the others. To our knowledge, Bing is the only search engine that
% provides a public API, so we would have to coordinate with other search
% engines to make this happen.
