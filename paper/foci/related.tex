
\section{Related work}
Existing block lists have several limitations. First, some of these
block lists have been curated by organizations that study Internet
censorship, but these lists are now
outdated~\cite{chinadigitaltimes,oni}. Other lists have been automatically
generated by systems similar to ours but are not publicly
available~\cite{sfakianakis2011censmon}. There are also systems that
create block lists through crowd-sourcing, but are unable to
automatically detect newly censored websites~\cite{herdict,
greatfire}. Finally, the Citizen Lab block list focuses on popular
websites--such as social media, Western media outlets, and VPN
providers--but not websites pertaining to Chinese
culture~\cite{citizenlab:block}.

Other systems use block lists to determine \textit{how} censorship
works, but they do not create more block lists. For example, Pearce
et al. proposed a system that uses block lists to measure how DNS
manipulation works on a global scale by combining DNS queries with AS
data, HTTPS certificates, and more~\cite{pearce2017global}. Pearce
et al. also built Augur, a system that utilizes TCP/IP side channels
between a host and a potentially censored website to determine whether
or not they can communicate with each
other~\cite{pearce2017augur}. Furthermore, Burnett et al. proposed
Encore, a system that utilizes cross-origin requests to potentially
censored websites to \textit{passively} measure how censorship works
from many vantage points~\cite{burnett:encore}. Lastly, several
platforms have been proposed that crowd-source censorship measurements
from users by having them install custom software on their
devices~\cite{razaghpanah2016exploring, ooni:about, iclab}.
